%%%%%%%%%%%%%%%%%%%%%%%%%%%%%%%%%%%%%%%%%%%%%%%%%%%%%%%%%%%%%%%%%%%%%%%%%%%%%%%%
%% commands.tex
%% 2009.01.09 - tedmunds
%%
%% This file is a header file used to specify document-wide macros.
%% This file is included by the template's main header file (./header.tex)
%%
%%%%%%%%%%%%%%%%%%%%%%%%%%%%%%%%%%%%%%%%%%%%%%%%%%%%%%%%%%%%%%%%%%%%%%%%%%%%%%%%

%% Create a command for typesetting the BibTeX logo.
%% The \providecommand command is used (instead of newcommand) incase some
%% package already provides the logo.
\providecommand{\BibTeX}{{\sc Bib}\TeX}

%% Define a \termdefinition command that standardizes the typesetting of newly
%% introduced terminology.
\newcommand{\termdefinition}[1]{\emph{#1}}

%% Define a \vectorize command that will typeset the argument in a bold-face
%% font, indiciative of a non-scalar variable.
\newcommand{\vectorize}[1]{\ensuremath{\mathbf#1}}

%% Define a \degree command as a shortcut to produce the degrees symbol.
%% (We use \providecommand so that if the author is using a package that
%% already provides a nice degree symbol, it won't be replaced).
\providecommand{\degree}{\ensuremath{^\circ}}

%% If you want to use unit notation without worrying about whether you are in
%% math-mode, you can define a nice consistent typesetting
\newcommand{\newtons}{\ensuremath{\mathrm{N}}}
\newcommand{\metres}{\ensuremath{\mathrm{m}}}
\newcommand{\millimetres}{\ensuremath{\mathrm{mm}}}%

%% There does not seem to be a standard command for rendering the mathematical
%% operation of taking the minimum of two quantities.
\newcommand{\minimum}[2]{\ensuremath{{\mathrm {min}}(#1,#2)}}%

%% If you frequently write the name of some device or technique, you might want
%% to define a macro for it so that you can easily add a trademark notice or
%% change the capitalization whenever the originator rebrands.
\newcommand{\Phantom}{PHANTOM}%

%% Your algorithm is always better than the na�ve algorithm, right?
\newcommand{\naive}{na\"{\i}ve}
\newcommand{\Naive}{Na\"{\i}ve}


\newcommand{\argmax}{\operatornamewithlimits{argmax}}
\newcommand{\argmin}{\operatornamewithlimits{argmin}}
\newcommand{\larrow}{\operatornamewithlimits{\leftarrow}}
\newcommand{\Rmax}{R_{\rm max}}
\newcommand{\Rmin}{R_{\rm min}}
\newcommand{\Vmax}{V_{\rm max}}
\newcommand{\Vmin}{V_{\rm min}}

\newcommand{\vchoose}[2]{\frac {\left(\sum_#2 #1_#2\right)!}{\prod_#2 #1_#2!}}
\newcommand{\vgamma}[2]{\frac {\Gamma\left(\sum_#2 #1_#2\right)}{\prod_#2 \Gamma(#1_#2)}}
\newcommand{\vgammainv}[2]{\frac {\prod_#2 \Gamma(#1_#2)}{\Gamma\left(\sum_#2 #1_#2\right)}}

\newcommand{\prior}{\textsf}
\newcommand{\alg}{\textbf}
\newcommand{\env}{\emph}
